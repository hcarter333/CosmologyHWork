
\documentclass[prb,preprint]
{revtex4-1} 
% The line above defines the type of LaTeX document.
% Note that AJP uses the same style as Phys. Rev. B (prb).

% The % character begins a comment, which continues to the end of the line.

\usepackage{amsmath}  % needed for \tfrac, \bmatrix, etc.
\usepackage{amsfonts} % needed for bold Greek, Fraktur, and blackboard bold
\usepackage{graphicx} % needed for figures
\newcommand{\PRLsep}{\noindent\makebox[\linewidth]{\resizebox{0.8888\linewidth}{2pt}{$\bullet$}}\bigskip}

\begin{document}

% Be sure to use the \title, \author, \affiliation, and \abstract macros
% to format your title page.  Don't use lower-level macros to  manually
% adjust the fonts and centering.

\title{Cosmology Homework I Due 2015/02/06}
% In a long title you can use \\ to force a line break at a certain location.

\author{Hamilton B. Carter}
\email{hcarter333@tamu.edu} % optional
% optional second address
% If there were a second author at the same address, we would put another 
% \author{} statement here.  Don't combine multiple authors in a single
% \author statement.
\affiliation{Department of Physics, Texas A\&M University, College Station, TX 77843}
% Please provide a full mailing address here.

% See the REVTeX documentation for more examples of author and affiliation lists.

\date{\today}

%\begin{abstract}


%\end{abstract}
% AJP requires an abstract for all regular article submissions.
% Abstracts are optional for submissions to the "Notes and Discussions" section.




\maketitle % title page is now complet

%\newpage
%\section{Board 1}

%\begin{figure}[h!]
%\centering
%\includegraphics[width=5in]{board1_2014_06_24.jpg}
% Notice the width specification.  Photographs should normally have a
% resolution of approximately 300 pixels per inch when printed, that is,
% a total width of about 1000 pixels for a photo to be printed one column
% wide.  Note also that this included photo is in .jpg format even though 
% a .tiff version should be submitted for final production.
%\caption{Board 1)}
%\label{Board 1}
%\end{figure}
%\centerline{\bf EMII Homework I Due 2014/09/17}
%\bigskip


\textbf{1.}
\\
Because it's more fun and the methodology produces results similar to the given information for problem number 3, I'll follow Karapetoff\cite{karapetoff1}, (the article is also attached to the email that submitted this homework), in writing down the Lorentz transform as:
$\Lambda^\alpha_{\;\beta} = \begin{pmatrix}
cosh\left(u\right) & -sinh\left(u\right)\\
-sinh\left(u\right) & cosh\left(u\right)\\
\end{pmatrix} 
 $
\\
\\
Now, we can write down the transformation as 
\\
\\
$\begin{pmatrix}
dx^\prime\\
dt^\prime\\
\end{pmatrix} = \begin{pmatrix}
cosh\left(u\right) & -sinh\left(u\right)\\
-sinh\left(u\right) & cosh\left(u\right)\\
\end{pmatrix} 
\begin{pmatrix}
dx\\
dt\\
\end{pmatrix}
 $
\\
\\
or, 
\\
\\
$dx^\prime = dx\;cosh\left(u\right) - dt\;sinh\left(u\right)$
\\
$dt^\prime = dt\;cosh\left(u\right) - dx\;sinh\left(u\right)$
\\
\\
Now, if we calculate $ds^{\prime 2}$, we get
\\
\\
$ds{\prime 2} = dt^{\prime 2} - dx^{\prime 2}$
\\
$= dt^2\;cosh^2\left(u\right) + dx^2\;sinh^2\left(u\right) - dx^2\;cosh^2\left(u\right) - dt^2\;sinh^2\left(u\right)$
\\
$= dt^2 - dx^2$
\\
\\
So,
\\
\\
$ds^{\prime 2} = ds^2$
\\
\\
This leaves the question as to where we incorporated $\vec{v}$.  If we consider a simple event where a particle stays at its origin the primed frame, then, plugging in $0$ for $x^\prime$ in the transformation, 
\\
\\
$dx^\prime = dx\;cosh\left(u\right) - dt\;sinh\left(u\right)$
\\
\\
We get 
\\
\\
$\dfrac{dx}{dt} = tanh\left(u\right)$
\\
\\
or,
\\
\\
$u = atanh\left(v\right)$
\\
\\
Where it should be remembered that we have set $c = 1$.  
\\
\\
\PRLsep
\\
\newpage
\textbf{2.}
\\
Let's do this the clunky matrix way first, and then we'll go back and pick up the boost as rotation method if time allows.  First, write down the matrix representations of the two successive boosts.  We won't worry abou the order of multiplication since our intent is to show that the result commutes in the end anyway.  First, fill in the components of the two Lorentz transforms keeping in mind that 
\\
$\Lambda^i_{\; j} = \delta_{ij} + \dfrac{\gamma - 1}{v^2} v_i v_j$
\\
We'll need this for part 2.d. if not sooner
\\
$\begin{pmatrix}
\gamma_1 & -\gamma_1 v_1\\
-\gamma_1 v_1 & \gamma_1\\
\end{pmatrix} 
\begin{pmatrix}
\gamma_2 & -\gamma_2 v_2\\
-\gamma_2 v_2 & \gamma_2\\
\end{pmatrix} $
\\
$  = \gamma_1 \gamma_2 
\begin{pmatrix}
v_1 v_2 + 1 & -v_2 - v_1\\
-v_1 - v_2 & v_1 v_2 + 1\\
\end{pmatrix} $
\\
Now, if we just bludgeon through we get
\\
$\begin{pmatrix}
\gamma_2 & -\gamma_2 v_2\\
-\gamma_2 v_2 & \gamma_2\\
\end{pmatrix} 
\begin{pmatrix}
\gamma_1 & -\gamma_1 v_1\\
-\gamma_1 v_1 & \gamma_1\\
\end{pmatrix} $
\\
$  = \gamma_1 \gamma_2 
\begin{pmatrix}
v_1 v_2 + 1 & -v_2 - v_1\\
-v_1 - v_2 & v_1 v_2 + 1\\
\end{pmatrix} $
\\
They produce the same results and therefore commute.  Since we can always adjust our coordinate system so that the velocities lie along the x axis, this result holds for any direction of velocity as long as both boosts are parallel as stated in the problem.
\\
\PRLsep
\\
Now, for the fun ways.  The first obvious solution is just to state the four velocity transform we derived in class and in the notes.  We have for the transformation of a velocity $u$ to a frame moving in the same direciton with the velocity $v$ that
\\
$u^\prime = \dfrac{u_x - v}{1-u_x v}$
\\
\\
Suppose we had asked a differnt question, what if, of asking what the velocity transforms to in the moving frame, we asked what the velocity is if we boost the particle into the primed frame.  The answer would be
\\
$v_{lab} = \dfrac{u_x + v}{1+u_x v}$
\\
It is manifest that the expression commutes, and a quick glance ahead to the answer for 2b will show that it matches the clunkier result from working with matrices.
\\
\\
Let's look at it yet another way.  We keep alluding to the fact that Lorentz transforms are rotation in space-time.  Let's see if there's a natural way to actually treat them as such and if so, how that helps.  Again, the  simple place to start is wth the four velocity.
\\
\\
The 0 component of the four velocity is just $\gamma$.  Suppose we didn't set c to 1 in our expressions.  We'd have,
\\
\\
$\gamma = \dfrac{1}{\sqrt{1 - \dfrac{v^2}{c^2}}}$
\\
$= \dfrac{1}{\dfrac{1}{c}\sqrt{c^2 - v^2}}$
\\
$= \dfrac{c}{\sqrt{c^2 - v^2}}$
\\
\\
Those familiar with hyperbolic geometry will recognize this as the hyperbolic cosine of an angle where the hyptoenuse is $\sqrt{c^2 - v^2}$ and the adjacent side is $c$.
\\
\\
The space component of four velocity can be written as, (taking a bit more care to place the factors of c back where they go using 7.2 in L\&L),
\\
$\gamma v = \dfrac{cv}{\sqrt{c^2 - v^2}} = c\;sinh\left(\phi\right)$
\\
\\
Which is nothing more than a hyperbolic sine times a factor of c.  Here, the opposite side is $v$ and the hyptoenuse is of course the same.
\\
\\
To relate these quantities back to labortory velocity, we'd need to remember that $\gamma = \dfrac{dt}{d\tau}$ and that the spatial portion of four velocity is expressed as $u = \dfrac{d\vec{x}}{d\tau}$.  Now, if we divide the hyperbolic sine by the hyperbolic cosine we get, 
\\
\\
$\dfrac{c\;sinh\left(\phi\right)}{cosh\left(\phi\right)} = \dfrac{d\vec{x}}{d\tau} \dfrac{dt}{d\tau} = \dfrac{d\vec{x}}{dt} = tanh\left(\phi\right)$
\\
\\
Which means that lab velocity can be written as
\\
\\
$\dfrac{v}{c} = tanh\left(\phi\right)$
\\
\\
and our hyperbolic angle can be written as 
\\
\\
$\phi = atanh\left(\dfrac{v}{c}\right)$
\\
\\
We're getting somewhwere, we know that angles in the same plane of a rotaiton simply add.  Can the same thing work here?  We'll have to find out later.
\\
\\
\PRLsep
\\
\\
Now, we have to show that result above is still a Lorentz boost.
\\
Let's suppose it is a Lorentz boost and see if things work out.  We start with
\\
$\gamma_1 \gamma_2 
\begin{pmatrix}
v_1 v_2 + 1 & -v_2 - v_1\\
-v_1 - v_2 & v_1 v_2 + 1\\
\end{pmatrix} $
\\
which can be rewritten suggestively as
\\
$\gamma_3 
\begin{pmatrix}
1 & \dfrac{-v_2 - v_1}{v_1 v_2 + 1}\\
\dfrac{-v_1 - v_2}{v_1 v_2 + 1} & 1\\
\end{pmatrix} $
\\
where 
\\
$\gamma_3 = \gamma_1 \gamma_2 \left(v_1 v_2 + 1\right)$
\\
It looks like our new velocity should be 
\\
$v_3 = \dfrac{v_2 + v_1}{1 + v_1 v_2}$
\\
Let's see if this works out in terms of our new gamma.  I'll work with $\gamma^{-1}$ just to avoid working in denominators all day, and then I'll take the reciprocal of the result at the end.
\\
\\
$1 - v_3^2 = 1 - \dfrac{v_1^2 + 2v_1^2 v_2^2 + v_2^2}{1 + 2 v_1 v_2 + v_1^2 v_2^2}$
\\
\\
Setting terms over the same common denominator we get:
\\
\\
$= \dfrac{1 + 2 v_1 v_2 + v_1^2 v_2^2}{1 + 2 v_1 v_2 + v_1^2 v_2^2} - \dfrac{v_1^2 + 2v_1^2 v_2^2 + v_2^2}{1 + 2 v_1 v_2 + v_1^2 v_2^2}$
\\
\\
Simplifying gives:
\\
\\
$= \dfrac{1 - v_1^2 - v_2^2 + v_1^2 + v_2^2}{1 + 2 v_1 v_2 + v_1^2 v_2^2}$
\\
\\
Taking the square root gives
\\
\\
$\sqrt{1 - v_3^2} = \dfrac{\sqrt{1-v_1^2}\sqrt{1-v_2^2}}{1+ v_1 v_2}$
\\
\\
Taking the reciprocal, we get 
\\
\\
$\gamma_3 = \dfrac{1+ v_1 v_2}{\sqrt{1-v_1^2}\sqrt{1-v_2^2}} = \gamma_1 \gamma_2 \left(1+ v_1 v_2\right)$
\\
\\
which is the gamma factor required for the resultant velocity.
\\
\PRLsep
\\
\textbf{3.}
\\
\\
We need to find out if $ds^2$, the four-space line element is invariant under a given transformation.
The given transformation is 
\\
\\
$t' = \dfrac{1}{g}\left(e^{gz}\right)sinh\left(gt\right)$
\\
\\
$z' = \dfrac{1}{g}\left(e^{gz}\right)cosh\left(gt\right)$
\\
\\
So,
\\
\\
$dt' = \dfrac{1}{g}g\left(e^{gz}\right)sinh\left(gt\right)dz + \dfrac{1}{g}g\left(e^{gz}\right)cosh\left(gt\right)dt$
\\
\\
$dz' = \dfrac{1}{g}g\left(e^{gz}\right)cosh\left(gt\right)dz + \dfrac{1}{g}g\left(e^{gz}\right)sinh\left(gt\right)dt$
\\
\\
$dt^{\prime 2} = e^{2gz}sinh^2\left(gt\right)dz^2 + e^{2gz}cosh^2\left(gt\right)dt^2$
\\
\\
Where the mixed differentials have been omitted because they will cancel with the mixed terms from $dz^{\prime 2}$.
\\
\\
$dz^{\prime 2} = e^{2gz}cosh^2\left(gt\right)dz^2 + e^{2gz}sinh^2\left(gt\right)dt^2$
\\
\\
$ds^{\prime 2} = dt^{\prime 2} - dz^{\prime 2} = e^{2gz}sinh^2\left(gt\right)dz^2 + e^{2gz}cosh^2\left(gt\right)dt^2 - \left(e^{2gz}cosh^2\left(gt\right)dz^2 + e^{2gz}sinh^2\left(gt\right)dt^2\right)$
\\
\\
$= e^{2gz}\left(sinh^2\left(gt\right) - cosh^2\left(gt\right)\right)dz^2 + e^{2gz}\left(cosh^2\left(gt\right) - sinh^2\left(gt\right)\right)dt^2$
\\
\\
\framebox{$ds^{\prime 2} = e^{2gz}\left(dt^2 - dz^2\right)$}
\\
\\
\framebox{Consequently, $ds^{\prime 2}$ is not invariant in this metric.}
\\
\\
Just as a quick check, what if we had factored the $gz$ term back in as a phase to the $sinh$ and $cosh$ functions right away?  How would that have changed things?  Can the $e^{gz}$ even be factored back in as a phase?
\\
\\
$sinh$ can be written as 
\\
\\
$sinh\left(x\right) = \dfrac{e^x +e^{-x}}{2}$
\\
\\
Factoring the $e^{gz}$ back in gives us
\\
\\
$\dfrac{e^{gt + gz} +e^{-gt + gz}}{2}$
\\
\\
So, it looks like life is good.  We can't get back to a single $sinh$ or $cosh$ when you take the $gz$ back into the expression.
\\
\\
\textbf{Questions}
The metric given for this problem looks like the Davies and Birrell form of the Rindler coordinates.  However, if we write down Rindler coordinates from Rindler's "Special Relativity" book, (1960, p. 41), they are:
\\
\\
$t = \dfrac{1}{g}sinh\left(g \tau\right)$
\\
\\
$z = \dfrac{1}{g}cosh\left(g \tau\right) - \dfrac{1}{g}$
\\
\\
While the missing leading exponential term, (since it's not there), will not cause the final result to not be invariant in $ds^2$, other issues may.  Playing the same $dt^2 - dz^2$ trick as above, (even though we're not looking at the primed, but the unprimed frame), results in 
\\
\\
$ds^2 = dt^2 - dz^2 = cosh^2\left(g\tau\right)d\tau^2 - sinh^2\left(g\tau\right)d\tau^2$
\\
\\
or $ds^2 = d\tau^2$
\\
\\
This seems to indicate that Rindler coordinates maintain an invariance in $ds$ even though they incorporate acceleration and a non-inertial frame.  It would also seem to indicate that the metric given in this problem is not in fact the same as Rindler coordinates.
\\
\\
\PRLsep
\newpage
\textbf{4.}
\\
The line element for spherical coordinates is 
\\
\\
$ds^2 = dr^2 + r^2d\theta^2 + r^2in^2\theta d\phi^2$
\\
\\
The inverse metric can be obtained by inverting the following matrix
\\
\\
$\begin{pmatrix}
dr^2 & 0 ^ 0\\
0 & r^2d\theta^2 & 0\\
0 & 0 & r^2\;sin^2\;\theta\;d\phi^2\\
\end{pmatrix}$
\\
\\
This gives
\\
\\
$\begin{pmatrix}
\dfrac{1}{dr^2} & 0 ^ 0\\
0 & \dfrac{1}{r^2d\theta^2} & 0\\
0 & 0 & \dfrac{csc^2\theta}{r^2\;d\phi^2}\\
\end{pmatrix}$
\\
\\
When $r = R$, the two dimensional metric is 
\\
\\
$ds^2 = R^2\left(d\theta^2 + sin^2\theta d\phi^2\right)$
\\
\PRLsep
\newpage
\begin{thebibliography}{99}
% The numeral (here 99) in curly braces is nominally the number of entries in
% the bibliography. It's supposed to affect the amount of space around the
% numerical labels, so only the number of digits should matter--and even that
% seems to make no discernible difference.
% The issue number (3) in this citation is optional, because AJP's pagination 
% is by volume.

\bibitem{karapetoff1} Karapetoff, V., ``Restricted Theory of Relativity in Terms of Hyperbolic Functions of Rapidities", The American Mathematical Monthly, \textbf{43}, 70--82 (1936).  


\end{thebibliography}

% If your manuscript is conditionally accepted, the editors will ask you to
% submit your editable LaTeX source file.  Before doing so, you should move
% all tables and figure captions to the end, as shown below.  Tables come 
% first, followed by figure captions (with figure inclusions commented-out).
% Figures should be submitted as separate files, collected with the
% LaTeX file into a single .zip archive.

%\newpage   % Start a new page for tables

%\begin{table}[h!]
%\centering
%\caption{Elementary bosons}
%\begin{ruledtabular}
%\begin{tabular}{l c c c c p{5cm}}
%Name & Symbol & Mass (GeV/$c^2$) & Spin & Discovered & Interacts with \\
%\hline
%Photon & $\gamma$ & \ \ 0 & 1 & 1905 & Electrically charged particles \\
%Gluons & $g$ & \ \ 0 & 1 & 1978 & Strongly interacting particles (quarks and gluons) \\
%Weak charged bosons & $W^\pm$ & \ 82 & 1 & 1983 & Quarks, leptons, $W^\pm$, $Z^0$, $\gamma$ \\
%Weak neutral boson & $Z^0$ & \ 91 & 1 & 1983 & Quarks, leptons, $W^\pm$, $Z^0$ \\
%Higgs boson & $H$ & 126 & 0 & 2012 & Massive particles (according to theory) \\
%\end{tabular}
%\end{ruledtabular}
%\label{bosons}
%\end{table}

%\newpage   % Start a new page for figure captions

%\section*{Figure captions}

%\begin{figure}[h!]
%\centering
%\includegraphics{GasBulbData.eps}   % This line stays commented-out
%\caption{Pressure as a function of temperature for a fixed volume of air.  
%The three data sets are for three different amounts of air in the container. 
%For an ideal gas, the pressure would go to zero at $-273^\circ$C.  (Notice
%that this is a vector graphic, so it can be viewed at any scale without
%seeing pixels.)}

%\label{gasbulbdata}
%\end{figure}

%\begin{figure}[h!]
%\centering
%\includegraphics[width=5in]{ThreeSunsets.jpg}   % This line stays commented-out
%\caption{Three overlaid sequences of photos of the setting sun, taken
%near the December solstice (left), September equinox (center), and
%June solstice (right), all from the same location at 41$^\circ$ north
%latitude. The time interval between images in each sequence is approximately
%four minutes.}
%\label{sunsets}
%\end{figure}

\end{document}
